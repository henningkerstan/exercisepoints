% !TEX encoding = UTF-8 Unicode
% !TEX program = pdflatex
%
%% exam-n.tex
%% Example file showing how to use the exercisepoints package 
%% with the exam-n class.
%%
%% Copyright (c) 2019 Henning Kerstan
%
% This work may be distributed and/or modified under the
% conditions of the LaTeX Project Public License, either version 1.3
% of this license or (at your option) any later version.
% The latest version of this license is in
%   http://www.latex-project.org/lppl.txt
% and version 1.3 or later is part of all distributions of LaTeX
% version 2005/12/01 or later.
%
% This work has the LPPL maintenance status `maintained'.
% 
% The Current Maintainer of this work is Henning Kerstan.
%

\documentclass[draft]{exam-n}

% load the exercisepoints package with the customlayout option
\usepackage[customlayout]{exercisepoints}

% amssymb is needed for the math typesetting of the example below
\usepackage{amssymb}

% set up exercise layout (using the question environment of exam-n)
\AtBeginExercise{%
    \begin{question}{\currentexercisepoints}\textbf{\currentexercisetitle}~\\
}
\AtEndExercise{\end{question}}

% simultaneously use exercisepoint's "\points{x}" command (for counting)
% and exam-n's "\partmarks{x}" command (for typesetting); 
\makeatletter
\newcommand{\mypartmarks}{\@ifstar{\@mypartmarksstar}{\@mypartmarksnostar}}
\newcommand{\@mypartmarksstar}[1]{\partmarks*{#1}\points{#1}}
\newcommand{\@mypartmarksnostar}[1]{\partmarks{#1}\points{#1}}
\makeatother

\begin{document}

% use exercise environment instead of question environment in the document
\begin{exercise}[Calculus]
Determine the derivatives of the following functions.

% use \mypartmarks* instead of \partmarks*, analogously 
% use \mypartmarks instead of \partmarks
\part $f\colon \mathbb{R} \to \mathbb{R}$, $f(x) = x^2+2x+3$ \mypartmarks*{1}
\begin{solution}
    $f'\colon \mathbb{R} \to \mathbb{R}$, $f'(x) = 2x+2$
\end{solution}
\part $g\colon \mathbb{R} \to \mathbb{R}$, $g(x) = \exp(x^2)$ \mypartmarks*{3}
\begin{solution}
    $g'\colon \mathbb{R} \to \mathbb{R}$, $g'(x) = 2x \cdot \exp(x^2)$
\end{solution}
\end{exercise}

\end{document}